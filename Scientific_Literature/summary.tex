\documentclass[a4paper,12pt]{article}
\usepackage[utf8]{inputenc}
\usepackage[T1]{fontenc}

\usepackage{graphicx} 
\usepackage{caption}
\usepackage{subcaption}
\usepackage{booktabs} 
\usepackage{xcolor}
\usepackage{amsmath}
\usepackage{amsthm}
\usepackage{amssymb}
\usepackage{systeme}
\usepackage{hyperref}

\usepackage[top=3cm,bottom=3cm, left=2cm, right=2cm]{geometry}


\title{Summary of certain research articles on maternal mortality}
\date{}
\author{}

\begin{document}

\maketitle


\section{An Introduction to Maternal Mortality}

\textit{An Introduction to Maternal Mortality}, Nawal M Nour, Reviews in Obstetrics and Gynecology, 2008.\\
\url{https://pmc.ncbi.nlm.nih.gov/articles/PMC2505173/}

In 1987, the International Safe Motherhood Conference in Kenya established a goal of a 50\% reduction in maternal mortality by 2000, which was not achieved. In 2000 the UN issued 8 Millenium Development Goals (MDG), among which the fifth (MDG-5) aimed at a reduction of 75\% between 1990 and 2015. 

Main causes of maternal mortality:
\begin{itemize}
\item postpartum hemorrhage (24\%)
\item indirect causes, such as anemia, malaria or heart diseases (20\%)
\item infection (15\%)
\item unsafe abortion (13\%)
\item eclampsia (12\%)
\item obstructed labor (8\%)
\item ectopic pregnancy, embolism, anesthesia complications (8\%)
\end{itemize}

In resource-poor countries, MM is atributed to the 3 delays: delay in deciding to seek care (most births occur at home and a lack of skill of diagnosing complications leads to seeking care late), delay in reaching care in time (difficult access to a healthcare facility), delay in receiving adequate treatment (lack of technology).

In order to reduce MM rate, international organizations focus on:
\begin{itemize}
\item better care during the intra- and post-partum period, by providing better training, as well as adequate medication,
\item offering family planning services, leading to better contraception practices and thus preventing indirectly unsafe abortions, among others,
\item offering safer abortion options,
\item better antepartum care, although the effectiveness of this measure is not established. 
\end{itemize}

Other challenges:
\begin{itemize}
\item improving overall healthcare system,
\item improving data collection: civil registration of death is the most reliable way, but its accuracy is questionable and MM deaths are likely underreported,
\item uncertainty on what the best practice for reduction of MM is.
\end{itemize}

Of note: Bolivia, Brazil, China, Egypt, Morocco and Peru have apparently made a lot of progress towards achieving MDG-5. Those countries have implement programs targeting the interventions mentioned above, as well as empowering and educating women and improving their healthcare access.


\section{Maternal mortality: who, when, where, and why?}

\textit{Maternal mortality: who, when, where, and why}, Ronsmans \textit{et al}, The Lancet, 2006.\\
\url{https://www.thelancet.com/journals/lancet/article/PIIS0140-6736(06)69380-X/abstract?isEOP=true}

Some definitions:
\begin{itemize}
\item Maternal deaths: death of a pregnant woman or within 42 days of termination of pregnancy,
\item MM ratio: number of maternal deaths during given time period per 100.000 livebirths during sametime period. \textit{Most commonly used indicator, but captures the probability of dying for a pregnant woman, and hence is sometimes called obstetric risk}
\item MM rate: number of maternal deaths in given time period per 100.000 women of reproductive age, or woman-years of risk exposure, in same time period,
\item Lifetime risk of maternal death: probability of maternal death during a woman’s reproductive life, usually expressed in terms of odds
\item Proportionate mortality ration: maternal deaths as proportion of all female deaths of those of reproductive age—usually defined as 15–49 years—in a given time period
\item Some new indicators also take the one-year period after child birth into account.
\end{itemize}

An argument is made that MM ratio does not capture the reduction of MM due to declining fertility, unlike lifetime risk of maternal death. Given that fertility is globally declining fast, we might want to keep this in mind.

Historically, in some industrialized countries, reduction of MM is due to professional midwifery (late 19th century), and better access to hospital care (after WW2).

Thailand, Malaysia and Sri Lanka are three examples of a recent decline in MM, which is thought to be the result of investing in midwifery, free care, control and supervision of midwifery. Egypt and Honduras are on similar tracks. 

Differences in MM cannot only be explained by variations in economic growth. Vietnam and Sri Lanka have much lower MM than Yemen or Ivory Coast. Within countries, there may also be inequalities: rich/poor (reasons are unclear, as the uptake of delivery services is likely not the only explanation), urban/rural (e.g. Egypt had a more than twice higher MM ratio in the nomadic Frontier region than in the Metropolitan region), which can be explained at least in part to access to obstetric care. In urban areas, high prevalence of HIV or unsafe abortion can also lead to high MM.

\section{Estimates of maternal mortality worldwide between 1990 and 2005: an assessment of available data}

\textit{Estimates of maternal mortality worldwide between 1990 and 2005: an assessment of available data}, Hill \textit{et al}, The Lancet, 2007.\\
\url{https://www.thelancet.com/journals/lancet/article/PIIS0140-6736(07)61572-4/abstract}

I was not able to get access to this article, but it apparently proposes a method to produce comparable estimates despite data weaknesses. 

\section{Recent Increases in the U.S. Maternal Mortality Rate}

\textit{Recent Increases in the U.S. Maternal
Mortality Rate}, MacDorman \textit{et al}, Obstetrics \& Gynecology, 2016.\\
\url{https://journals.lww.com/greenjournal/fulltext/2016/09000/Recent_Increases_in_the_U_S__Maternal_Mortality.6}

Similar starting point than the article above: weakness of data makes it difficult to extract trends. Here, the authors tackle this problem in the US and "correct" the data state by state to show that MM rate increases in the US.

\textit{I am not sure that right now these methods are going to be useful/applicable to our case, but it is good to know that they exist and might serve as inspiration along the way.}

\section{Global, regional, and national levels of maternal mortality,
1990–2015: a systematic analysis for the Global Burden of
Disease Study 2015}

\textit{Global, regional, and national levels of maternal mortality,
1990–2015: a systematic analysis for the Global Burden of
Disease Study 2015}, GBD 2015 Maternal Mortality Collaborators, The Lancet, 2016.\\
\url{https://www.thelancet.com/journals/lancet/article/PIIS0140-6736(16)31470-2/fulltext}

In 2015, 17 Sustainable Development Goals (SDG) were adopted, as a continuation of the spirit of the MDGs. SDG 3.1 sets a target to reduce MM ratio to less than 70 by 2030 and SDG 3.7 calls for universal access to reproductive and sexual health services (which was also a secondary target of MDG-5 adopted in 2005).

GBD (Global Burden of Disease) carried out this study to assess the success of MDG-5, but the methods are different than WHO (\textit{could be another interesting data set?}). 

\section{Trends in maternal mortality 2000 to 2020}

\textit{Trends in maternal mortality
2000 to 2020}, WHO, 2023.\\
\url{https://www.who.int/publications/i/item/9789240068759}

This is the newest iteration of a study that was previously published in 2008, 2010, 2013 and 2015 (and maybe other iterations?). 

\section{A global analysis of the determinants of maternal health and transitions in maternal mortality}

\textit{A global analysis of the determinants of maternal health and transitions in maternal mortality}, J.P. Souza \textit{et al.}, The Lancet Global Health, 2024.\\
\url{https://www.thelancet.com/journals/langlo/article/PIIS2214-109X(23)00468-0/fulltext}

This very recent study uses datasets from the WHO Study. 

Since 2016, except in Australia, New Zealand and Central- and South Asia, the maternal mortality ratio (MMR) has not decreased, which eventually might compromise the SDG goals. Moreover, 121 out of 185 analysed countries have been in the same maternal mortality stage for 20 years. Why is that?

\textit{Note: Stage 1: MMR $\geq$500, Stage 2: 300$\leq$MMR$\leq$ 500, Stage 3: 100$\leq$MMR$\leq$ 300, Stage 4: MMR$\leq$ 100}.

The two most effective tools to reduce MM has been so far 1) access to quality-assured maternal health services, 2) prevention of undesired pregnancies. However, more and more evidence suggests that promoting overall health and wellbeing across the life course might be the next necessary step in reducing MM. More precisely, the study identifies both distal (e.g., socio-economic, cultural, and environmental factors) and proximal (e.g., health behaviors, access to care) determinants that impact maternal health outcomes.

The authors then propose the concept of maternal mortality transition, which describes how much the advances in social developments counteract the determinants above. 

Health-system indicators (e.g. universal health coverage, skilled birth attendance) correlate strongly with MM levels, whilst other social indicators (e.g. Gini index, Human development index) show a lesser correlation (see Table 3). Caesearan section deserves further investigation.

\section{Global causes of maternal death: a WHO systematic analysis}

\textit{Global causes of maternal death: a WHO systematic analysis}, L. Say \textit{et al.}, The Lancet Global Health, 2014.\\
\url{https://www.thelancet.com/journals/langlo/article/PIIS2214-109X(14)70227-X/fulltext}

This study analyzes the causes of maternal death by looking at 417 datasets from 115 countries, totalizing 60799 deaths.

Table 1 estimates the percentage of different casuses of death depending on the region. Worldwide the causes of death rank (with some uncertainty):
\begin{enumerate}
\item Indirect causes ($27.5\%$) (\textit{e.g.} medical disorders, HIV-related maternal deaths, and other causes)
\item Haemorrhage ($27.1\%$)
\item Hypertension ($14.0\%$)
\item Sepsis ($10.7\%$)
\item Other direct causes ($9.6\%$) (\textit{i.e.} the remaining causes)
\item Abortion ($7.9\%$)
\item Embolism ($3.2\%$)
\end{enumerate} 

The rest of the article goes more into the details of the statistics, which are summarized in the different tables.

  




\end{document}

